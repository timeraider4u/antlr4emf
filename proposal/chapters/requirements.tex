% !TeX encoding = UTF-8
% !TeX spellcheck = en_US
\section{Requirements}
All requirements for the ''ANTLR4EMF'' project
are described in this section.
Each of them is rated based on the estimated effort
and on the importance.

\subsection{Basics}
ANTLR4EMF should be implemented as an Eclipse plug-in
which uses the Eclipse Modeling framework and 
exactly one stable library version of ANTLR version 4
(the latest stable release?).
It should be possible to use any ANTLR version 4
grammar file and provide this as the only 
required input to ANTLR4EMF. 
ANTLR4EMF will then generate a meta-model,
a gen-model, a lexer/parser and a 
visitor/listener for mapping parsed
grammar elements to meta-model elements
automatically.
So it should be pretty much like Xtext,
but without using a dedicated DSL file.
\subsubsection{Inputs}
\begin{itemize}
	\item .g4 - ANTLRv4 grammar specification file
\end{itemize}
\subsubsection{Expected outputs}
\begin{itemize}
	\item .xmi - Ecore meta-model
	\item .genmodel - Ecore gen-model for Ecore meta-model
	\item A Java-program that expects a file, which confirms to the
	ANTLRv4 grammar specification, as its input and writes an
	EMF model (.xmi file - file name as another argument?)
	to a file which confirms to the Ecore meta-model.
\end{itemize}
\subsubsection{WP meta-model generation}
\begin{itemize}
	\item For each Non-Terminal-Node a corresponding EClass 
	should be generated.
	\item For each TerminalNode a corresponding EString
	attribute in an EClass should be generated.
	\item Cardinality conversion should be easy:
	\begin{itemize}
		\item (no cardinality) turns into lower-bound 1 and upper-bound 1
		\item ? turns into 0 \dots 1
		\item + turns into 1 \dots -1
		\item * turns into 0 \dots -1
	\end{itemize}
	\item There are several pitfalls/problems with this task:
	\begin{itemize}
		\item The visitor/listener which is generated by ANTLR automatically
		does not store any information about the index of a TerminalNode.
		E.g., when we have {\it A: KW\_PUBLIC KW\_STATIC $|$ KW\_STATIC KW\_PUBLIC}
		we can not distinguish between these two versions without going
		down to the tokens layer (and compare the node positions in the 
		TokenStream).
		\item The visitor/listener which is generated by ANTLR automatically
		does not store any information about indexes of combined nodes.
		E.g., when we have {\it A: ((B C) $|$ (B D))+}
		and we have a sequence like {\it BC BD BC}
		we only get lists of Bs, lists of Cs and lists of Ds but we
		can not re-construct in which order the Cs and Ds occurred without
		going down to the tokens layer (and compare the node positions 
		in the TokenStream).
	\end{itemize} 
\end{itemize}
\subsubsection{Effort}
The effort may vary for the different work-packages.
Some might be easier and some more difficult.
But in summary, it should be average programming work
with a little bit thinking required.
\subsubsection{Importance}
This requirement with its corresponding work-packages
is a must-have. Without it the whole project is useless.

\subsection{Recursion-to-Non-recursion conversion}
It might be interesting to be able to convert
statements of the form {\it A: A B $|$ A C}
into {\it A: (B $|$ C)+} automatically. 
If this is a conversion of the .g4 files itself
or happens at a later point during parsing does not
matter.
\subsubsection{Inputs}
\begin{itemize}
	\item .g4 - ANTLRv4 grammar specification file
	(which might contain recursive rules)
\end{itemize}
\subsubsection{Expected outputs}
\begin{itemize}
		\item .g4 - ANTLRv4 grammar specification file
		(without recursive rules)
\end{itemize}
\subsubsection{Effort}
This might be tricky to develop an algorithm but
should be easy to implement once such an algorithm is found.
Maybe there exists already an implementation / a solution.
\subsubsection{Importance}
This is considered optional. But of course it would be nice
for real-world examples, as most grammars are written in
a recursive way.

\subsection{Model-walker}
As grammar rules for a programming language
are represented as a tree, it is possible
to walk through these elements. It would be 
nice if ANTLR4EMF also could generate a
walker for the EMF model (which is based
on the generated Ecore meta-model).
\subsubsection{Inputs}
\begin{itemize}
	\item .g4 - ANTLRv4 grammar specification file
\end{itemize}
\subsubsection{Expected outputs}
\begin{itemize}
	\item Tree-walker and visitor/listener template
	interfaces and classes for visiting EMF
	meta-model elements.
\end{itemize}
\subsubsection{Effort}
This might be easy to implement.
\subsubsection{Importance}
This is considered optional. But of course it would be nice
for real-world examples, as it improves a developers life 
a lot.
