% !TeX encoding = UTF-8
% !TeX spellcheck = en_US
\section{Existing T2M software - the good, the bad and the ugly}

Before exploring the requirements of this project, it will be a good idea to 
take a deeper look on the existing software with its advantages and 
problems. This will make the project scope and the requirements of this
proposal much clearer.
As the Eclipse Modeling Framework (EMF) with its Ecore meta-model is
a de-facto standard, only Eclipse plug-in projects will be described in this
section.

\subsection{EMFText}
''EMFText is a tool for defining textual syntax for Ecore-based metamodels''
\cite{Heidenreich2011model}.
It can be used for creating DSLs or parsing of GPLs, like Java in the Jamopp project
\cite{Heidenreich2009closing}.
EMFText uses ANTLR in version three under the hood.
Although it is built on top of it, ANTLR grammars can not be used one-to-one.
Instead, a dedicated DSL is used for the mapping of grammar elements to Ecore
elements.
It does not generate any
meta-model on itself but requires the users to provide it with one by themselves.
\\ \ \\
EMFText does not support ANTLR version 4 yet \cite{ANTLR4_EMFText}.

\subsection{Xtext}
Xtext\cite{Bettini2013implementing} is another project built on top of ANTLR version three.
It uses its own DSL for parsing of grammar elements and mapping them to meta-model elements.
One advantage of Xtext compared to EMFText is that it can generate an Ecore model automatically
based on these DSL description files. But it is also possible to re-use an existing meta-model
and just specify the parsing structure and the mapping. Xtext uses a dialect of ANTLR 3 but does 
not support all ANTLR3 language features (e.g., semantic predicates, which are needed
by C/C++ in ANTLR3, are not available).
\\ \ \\
Xtext does not support ANTLR version 4 yet \cite{ANTLR4_XText} and it is questionable
if it will ever support version four.
Xtext also does not use the latest release of the version three
branch, ANTLR 3.5, either. 

\subsection{Summary}
To summarize, both tools do not support ANTLR version 4.
The previous branch API version three is now not maintained anymore and
does not receive any updates upstream. 
Version 4 has the advantage that it uses another parsing algorithm and,
therefore, does not require syntactic or semantic predicates anymore
\cite{Parr2013definitive}\cite{parr1994adding}\cite{Parr2011ll}
(which are necessary for parsing of GPLs like C/C++ otherwise).
ANTLRv4 has the advantages that, first, there exists an awesome 
collection of grammar specifications for this version already at 
\cite{grammars_v4} on GitHub which seems to have
an active user and developer community. 
And, second, ANTLRv4 does provide a better separation of concerns
between grammar specifications and processing of parse-trees
by automatically generating visitor and listeners classes.

