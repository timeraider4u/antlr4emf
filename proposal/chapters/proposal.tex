% !TeX encoding = UTF-8
% !TeX spellcheck = en_US

Software developers and engineers often have to deal with 
modernization of legacy code
as part of the maintenance 
life cycle phase for software products. 
MDE (Model-driven engineering) promises to support software
builders with approaches, models, tools and processes to 
simplify this task by using automation.
The scientific term for
migration of software with the help of MDE 
is called 
''Model-Driven Software
Modernization'' (MDSM)\cite{Kowalczyk2009model}.
When the source of an application should be converted from 
one programming language to another, e.g. from Delphi to Java,
it is first necessary to import the existing code from text files
into a model which has a meta-model of the input 
programming language.
This first step is called Text-to-Model (T2M).
There exist domain-specific language (DSL) tools
which can be used or extended for simpler general-purpose
programming languages.
But when it comes to dealing with complex grammars, these tools
reach their limits. 
In \cite{Izquierdo2014extracting},
the authors discuss the existing problems, evaluate various existing
approaches and propose a new language to bridge the gap between
grammar-ware and Model-Driven Development (MDD).
\\ \ \\
But why invent another language. 
Is it really necessary?
With the advent of compiler-generators,
Yacc/Bison\cite{Bison},
JavaCC, 
Coco/R\cite{COCOR} and 
ANTLR\cite{Antlr}
for example,
it has gotten easier then ever to describe 
grammars in an explicit, programmatic form.
Collections of grammar files for different
programming languages, like \cite{Grammarzoo} and 
\cite{grammars_v4}, have arisen.
Wouldn't it be nice if they could just be used,
as they are, for
T2M projects?
